\documentclass{mwrep}

% Polskie znaki
\usepackage{polski}
\usepackage[polish]{babel}
\usepackage[utf8]{inputenc}
\usepackage[T1]{fontenc}
\usepackage[utf8]{luainputenc}
\usepackage{lmodern}
\usepackage{indentfirst}

% Strona tytułowa
\usepackage{pgfplots}
\usepackage{siunitx}
\usepackage{paracol}
\usepackage{gensymb}
\usepackage{afterpage}

% Pływające obrazki
\usepackage{float}
\usepackage{svg}
\usepackage{graphicx}

% table of contents refs
\usepackage{hyperref}
\usepackage{cleveref}
\usepackage{booktabs}
\usepackage{listings}
\usepackage{placeins}
\usepackage{xcolor}

\usetikzlibrary{pgfplots.groupplots}
\sisetup{detect-weight,exponent-product=\cdot,output-decimal-marker={,},per-mode=symbol,binary-units=true,range-phrase={-},range-units=single}
\definecolor{szary}{rgb}{0.95,0.95,0.95}
%konfiguracje pakietu listings
\lstset{
	backgroundcolor=\color{szary},
	frame=single,
	breaklines=true,
}
\lstdefinestyle{customlatex}{
	basicstyle=\footnotesize\ttfamily,
	%basicstyle=\small\ttfamily,
}

\lstdefinestyle{customjs}{
  keywords={typeof, new, true, false, catch, function, return, null, catch, switch, var, if, in, while, do, else, case, break},
  keywordstyle=\color{blue}\bfseries,
  ndkeywords={class, export, boolean, throw, implements, import, this},
  ndkeywordstyle=\color{darkgray}\bfseries,
  identifierstyle=\color{black},
  sensitive=false,
  comment=[l]{//},
  morecomment=[s]{/*}{*/},
  commentstyle=\color{purple}\ttfamily,
  stringstyle=\color{red}\ttfamily,
  morestring=[b]',
  morestring=[b]",
  basicstyle=\footnotesize\ttfamily,
  extendedchars=true,
  showstringspaces=false,
}

\lstdefinestyle{custompython}{
keywords={while, if, not, case, import, True},
keywordstyle=\color{blue}\bfseries,
basicstyle=\footnotesize\ttfamily
}

\lstdefinestyle{customlua}{
  language         = {[5.0]Lua},
  basicstyle       = \ttfamily,
  showstringspaces = false,
  upquote          = true,
}

\newcommand\blankpage{%
    \null
    \thispagestyle{empty}%
    \addtocounter{page}{-1}%
	\newpage}
	
\lstdefinestyle{customc}{
	breaklines=true,
	frame=tb,
	language=C,
	xleftmargin=0pt,
	showstringspaces=false,
	basicstyle=\small\ttfamily,
	keywordstyle=\bfseries\color{green!40!black},
	commentstyle=\itshape\color{purple!40!black},
	identifierstyle=\color{blue},
	stringstyle=\color{orange},
}
\lstdefinestyle{custommatlab}{
	%basicstyle=\fontsize{11}{13}\selectfont\ttfamily,
	captionpos=t,
	breaklines=true,
	frame=tb,
	xleftmargin=0pt,
	language=matlab,
	showstringspaces=false,
	%basicstyle=\footnotesize\ttfamily,
	basicstyle=\scriptsize\ttfamily,
	keywordstyle=\bfseries\color{green!40!black},
	commentstyle=\itshape\color{purple!40!black},
	identifierstyle=\color{blue},
	stringstyle=\color{orange},
}

%wymiar tekstu
\def\figurename{Rys.}
\def\tablename{Tab.}

%konfiguracja liczby p�ywaj�cych element�w
\setcounter{topnumber}{0}%2
\setcounter{bottomnumber}{3}%1
\setcounter{totalnumber}{5}%3
\renewcommand{\textfraction}{0.01}%0.2
\renewcommand{\topfraction}{0.95}%0.7
\renewcommand{\bottomfraction}{0.95}%0.3
\renewcommand{\floatpagefraction}{0.35}%0.5

\SendSettingsToPgf
\title{\bf Prezentacja i omówienie architektury modułu ESP8266 \vskip 0.1cm}
\author{Jakub Sikora}
\date{\today}
\pgfplotsset{compat=1.15}	
\begin{document}
\frenchspacing
\pagestyle{uheadings}

\makeatletter
\renewcommand{\maketitle}{\begin{titlepage}
		\begin{center}{
				\LARGE {\bf Politechnika Warszawska}}\\
            \vspace{0.4cm}
            \leftskip-0.9cm
            {\LARGE {\bf \mbox{Wydział Elektroniki i Technik Informacyjnych}}}\\
            \vspace{0.2cm}
            {\LARGE {\bf \mbox{Instytut Automatyki i Informatyki Stosowanej}}}\\
            
            \vspace{5cm}
            \leftskip-2.2cm
			{\bf \Huge \mbox{Zaawansowane architektury procesorów} \vskip 0.1cm}
		\end{center}
		\vspace{0.1cm}

		\begin{center}
			{\bf \LARGE \@title}
		\end{center}

		\vspace{9cm}
		\begin{paracol}{2}
			\addtocontents{toc}{\protect\setcounter{tocdepth}{1}}
			\subsection*{Autor:}
			\bf{ \Large{ \noindent\@author \par}}
			\addtocontents{toc}{\protect\setcounter{tocdepth}{2}}

			\switchcolumn \addtocontents{toc}{\protect\setcounter{tocdepth}{1}}
			\subsection*{Prowadzący:}
			\bf{\Large{\noindent \mbox{mgr inż. Grzegorz Mazur}}}
			\addtocontents{toc}{\protect\setcounter{tocdepth}{2}}

		\end{paracol}
		\vspace*{\stretch{6}}
		\begin{center}
			\bf{\large{Warszawa, \@date\vskip 0.1cm}}
		\end{center}
	\end{titlepage}
}
\makeatother
\maketitle
\blankpage
\tableofcontents

\chapter{Wstęp}
W niniejszym sprawozdaniu została zaprezentowana architektura modułu ESP8266. W celu 
zapoznania się z modułem, został przygotowany projekt czujnika temperatury komunikujący
się z serwerem HTTP, uruchamianym na komputerze osobistym.\\

W drugim rozdziale przedstawiona została architektura modułu, jego budowa, układ pamięci
oraz interfejsy zewnętrzne. W trzecim rozdziale przedstawiona została architektura 
oprogramowania napisana w języku C, wraz z opisem wykorzystanego środowiska programistycznego.
Rozdział czwarty opowiada już o samym projekcie czujnika, komunikującego się przez Wi-Fi
z komputerem osobistym. Piąty rodział przedstawia inne możliwości wykorzystywania modułu
w rozwiązaniach wbudowanych, bez korzystania z języka C.

\chapter{Omówienie architektury modułu ESP8266}
\label{omowienie_arch}
\chapter{Organizacja oprogramowania}
\label{organizacja_opr} 


\section{ESP8266 RTOS SDK}
\label{sdk}
W ramach projektu, za stroną producenta, zdecydowałem się na skorzystanie z 
szkieletu aplikacyjnego \verb+ESP8266_RTOS_SDK+, wykorzystującego FreeRTOS.
Takie podejście pozwala na pisanie prostych aplikacji wielozadaniowych. Dodatkowo, 
szkielet udostępnia rozbudowane API, które znacznie upraszcza korzystania z peryferiali oraz sieci Wi-Fi. 

\subsection{Instalacja środowiska}
\label{instalacja}
Dużą zaletą środowiska \verb+ESP8266_RTOS_SDK+ przygotowanego przez firmę \textit{Espressif}
jest prostota instalcji i korzystania z niego. Środowisko można pobrać ze strony producenta
lub z publicznego repozytorium znajdującego się w serwisie \verb+github.com+. Po instalacji
i ustawieniu koniecznych zmiennych środowiskowych, można przystąpić do wgrania programu.

\subsection{Tworzenie programu}
\label{tworzenie programu}
W celu stworzenia własnego aplikacji na moduł \verb+ESP8266+ należy skopiować folder 
\verb+project_template+ do wybranego folderu i w pliku \verb+user_main.c+ zapisać swój kod
źródłowy. Struktura przykładowego programu została przedstawiona w \ref{example_C}.

\subsection{Kompilacja i wgranie programu}
\label{kompilacja}
Środowisko znacząco upraszcza pracę nad aplikacjami, ponieważ dostarcza plik \verb+Makefile+
który samemu zajmuje się kompilacją i wgrywaniem programu. Z poziomu programisty, należy jedynie
wywołać polecenie \texttt{make flash}, które kolejno wywoła kompilator \verb+xtensa-lx106-elf-gcc+
a następnie narzędzie do wgrywania \verb+esptool.py+ napisane w języku Python.

W przypadku zwykłych modułów, konieczne jest ręczne ustawienie określonych pinów GPIO w taki sposób
aby przy resecie urządzenia przeszło ono w tryb UART, pozwalający programować pamięc Flash.
Po wgraniu programu, należy zresetować urządzenie, tak aby uruchomić bootloader w trybie FLASH, 
pozwalający na normalną pracę urzadzenia. Oprócz tego, istnieje jeszcze tryb SD, umożliwiający 
uruchomienie systemu z karty SD. Sposób ustawienia pinów został przedstawiony w tabeli 
\ref{tabela_trybow}

\begin{table}[H]
    \centering
    \begin{tabular}{lccc}
    \hline
          & \multicolumn{1}{l}{GPIO0} & \multicolumn{1}{l}{GPIO2} & \multicolumn{1}{l}{GPIO15} \\ \hline
    UART  & 0                         & 1                         & 0                          \\
    FLASH & 1                         & 1                         & 0                          \\
    SD    & x                         & x                         & 1                         
    \end{tabular}
    \caption{Porównanie pinów badanych przez bootloader podczas uruchamiania systemu}
    \label{tabela_trybow}
\end{table}
\FloatBarrier
\newpage
W przypadku modułu ESP8266-EVB użyczonego z zasobów Instytutu Informatyki w ramach tego projektu,
producent umieścił na płytce przycisk pozwalający na ustawienie omawianych pinów
w odpowiedni sposób tak aby przy włączeniu urządzenia przy wciśniętym przycisku, wchodził
on w tryb UART.

\begin{figure}[H]
	\centering
    \includegraphics[width=8cm]{./images/ESP8266-EVB.jpg}
    \caption{Płytka ESP8266-EVB firmy OLIMEX}
	\label{esp8266-evb}
\end{figure}
\FloatBarrier

W przypadku finalnie wykorzystanego przeze mnie modułu NodeMCU w wersji 3 z modułem
ESP8266, nie było potrzeby ręcznego przestawiania trybu pracy. Wszystkim tym zajmowało
się program do wgrywania kodu. Znacznie ułatwiało i przyspieszało to pracę nad projektem.
Płytka ta jest zasilana z kabla USB, który podpięty do komputera służył również do komunikacji
przez port szeregowy.

\begin{figure}[H]
	\centering
    \includegraphics[width=8cm]{./images/nodemcu.jpg}
    \caption{Wykorzystana płytka ewaluacyjna NodeMCU v3}
	\label{esp8266-nodemcu}
\end{figure}
\FloatBarrier

\subsection{Narzędzie \texttt{esptool.py}}
\label{esptool}
Narzędzie \texttt{esptool.py} umożliwia prgoramowanie modułu ESP8266. Zostało one 
napisane w języku Python w wersji 2. Aby zaprogramować płytkę należy wywołać 
program z odpowiednimi argumentami. Przykładowa komenda wygląda w następujący sposób:\\

\begin{lstlisting}[style=customc,
    frame=single,
    caption={Przykładowa komenda programująca moduł ESP8266},
    captionpos=b,
    label={esptool_basic}]
esptool.py --port COM4 write_flash 0x1000 my_app-0x01000.bin
\end{lstlisting}

Po komendzie \texttt{write\_{}flash} należy zapisać adres początkowy programu w 
pamięci Flash. Program umożliwia również konwersję plików \texttt{.elf} do postaci 
binarnej \texttt{.bin}. 

\texttt{esptool.py} pozwala również załadować program do pamięci RAM, tak jak to było
wspomniane w \ref{pamiec}. Aby w ten sposób wgrać program, należy skorzystać z polecenia 
\texttt{load\_{}ram}.

\section{Budowa przykładowego programu w języku C}
\label{example_C}
Przykładowy program oparty o FreeRTOS składa się zadań oraz kodu 
inicjalizującego. W pierwszej kolejnośći inicjalizowane są peryferia oraz inne struktury
systemowe, a następnie uruchamiane są zadania, które wykonują się w sposób współbieżny. \\

Na poniższym listingu została przedstawiona funkcja \texttt{app\_{}main} przykładowego programu
realizującego dwa zadania, od której moduł rozpoczyna pracę po restarcie.\\

\begin{lstlisting}[style=customc,
    frame=single,
    caption={Przykładowa funkcja \texttt{app\_{}main}},
    captionpos=b,
    label={esptool_basic}]
void app_main(){
  #include"freertos/FreeRTOS.h"
  #include"freertos/task.h"
  #include"esp_system.h"
  #include"driver/uart.h"
  #include"driver/adc.h"

  uart_init();
  initialize_adc();
    
  xTaskCreate(adc_task, "adc_task", 1024, NULL, 10, NULL);
  xTaskCreate(uart_task, "uart_task", 1024, NULL, 10, NULL);
}
\end{lstlisting}

Funkcja inicjuje wykorzystywany w programie UART oraz konwerter analogowo cyfrowy.
W dalszej kolejności, tworzone są zadania za pomocą funkcji \texttt{xTaskCreate}.
Jako pierwszy parametr pobiera ona wskaźnik na funkcję, która wykonuje dane zadanie.
\chapter{Projekt czujnika opartego o moduł ESP8266 napisany w języku C}
\label{projekt}

W celu prezentacji architektury modułu ESP8266 zdecydowałem się zaprogramować 
moduł w taki sposób aby mierzył on temperaturę i wysyłał tą informację do serwera
poprzez sieć Wi-Fi. Temperatura będzie mierzona analogowo za pomocą termistora, 
podłączonego do konwertera analogowo-cyfrowego. Czujnik będzie sygnalizował 
temperaturę za pomocą diody LED, której jasność będzie zwiększała się wraz z 
odczytem czujnika.

\section{Konfiguracja}
\label{projekt_konfiguracja}

\subsection{Interfejs szeregowy UART}
\label{projekt_uart}
W celu ułatwienia dalszej pracy, pracę na projektem rozpocząłem od konfiguracji
interfejsu UART, tak aby móc w prosty sposób wyswietlać na ekranie swojego komputera
informacje pochodzące od modułu. Inicjalizacją interfejsu UART zajmuje się funkcja
\verb+uart_init()+. \\

\begin{lstlisting}[style=customc,
    frame=single,
    caption={Konfiguracja interfejsu UART},
    captionpos=b,
    label={esptool_basic}]
void uart_init(){
  uart_config_t uart_config = {
    .baud_rate = 115200,
    .data_bits = UART_DATA_8_BITS,
    .parity    = UART_PARITY_DISABLE,
    .stop_bits = UART_STOP_BITS_1,
    .flow_ctrl = UART_HW_FLOWCTRL_DISABLE
  };
  uart_param_config(UART_NUM_0, &uart_config);
  uart_driver_install(UART_NUM_0, ECHO_BUFFER_SIZE*2, 
                      0, 0, NULL);
}
\end{lstlisting}

Funkcja ta w typowy sposób, znany z mikrokontrolerów z rodziny STM32, inicjuje interfejs UART
przy użyciu struktury przechowującej poszczególne parametry. W tym przypadku, struktura ta 
jest typu \verb+uart_config_t+. W niej zostały ustawione:\\
\begin{itemize}
    \item prędkość transmisji 
    \item ilość bitów danych
    \item bit parzystości
    \item ilość bitów stopu
    \item kontrola transmisji\\
\end{itemize} 

Rzeczywista konfiguracja interfejsu następuje po wywołaniu po sobie funkcji \verb+uart_param_config+ oraz
\verb+uart_driver_install+ z odpowiednim numerem interfejsu (w tym przypadku \verb+UART_NUM_0+).
\newpage

\subsection{Konwerter analogowo-cyfrowy}
\label{projekt_adc}

W podobny sposób został skonfigurowany konwerter analogo-cyfrowy do pomiaru napięcia odkładającego
się na termistorze. W tym celu przygotowana została funkcja \verb+initialize_adc+, która przygotowuje 
konwerter do pracy.

\begin{lstlisting}[style=customc,
    frame=single,
    caption={Konfiguracja ADC},
    captionpos=b,
    label={esptool_basic}]
void initialize_adc(){
  adc_config_t adc_config;
  adc_config.mode = ADC_READ_TOUT_MODE;
  adc_config.clk_div = 8;
  ESP_ERROR_CHECK(adc_init(&adc_config));
}
\end{lstlisting}

W tym przypadku, przygotowana została struktura typu \verb+adc_config_t+,
 w której ustawione zostały 
dwa parametry: tryb pracy oraz dzielnik częstotliwości taktowania konwertera. 
Tryb \verb+ADC_READ_TOUT_MODE+
ustawia konwerter w tryb pomiaru napięcia zewnętrznego. Rzeczywista 
konfiguracja konwertera następuję
w chwili wywołania funkcji \verb+adc_init+. Wywołanie tej funkcji 
makrem, pozwala na wypisanie informacji
o błędzie w przypadku niepowodzenia konfiguracji konwertera.


\subsection{Wi-Fi}
\label{projekt_wifi}

Zdecydowanie najciekawszym elementem do skonfigurowania była część 
obsługująca komunikację przez sieć Wi-Fi. W tym celu przygotowana została
funkcja \verb+wifi_init_sta+, która konfiguruje moduł w trybie \textit{Station}.\\

\begin{lstlisting}[style=customc,
    frame=single,
    caption={Konfiguracja Wi-Fi},
    captionpos=b,
    label={esptool_basic}]
void wifi_init_sta(){
  wifi_event_group = xEventGroupCreate();
  tcpip_adapter_init();

  wifi_init_config_t cfg = WIFI_INIT_CONFIG_DEFAULT();
  ESP_ERROR_CHECK(esp_wifi_init(&cfg));
  wifi_config_t wifi_config = {
    .sta = {
      .ssid = WIFI_SSID,
      .password = WIFI_PASS
    },
  };

  ESP_ERROR_CHECK(esp_wifi_set_mode(WIFI_MODE_STA));
  ESP_ERROR_CHECK(esp_wifi_set_config(ESP_IF_WIFI_STA, 
                                      &wifi_config));
  ESP_ERROR_CHECK(esp_wifi_start());
}
\end{lstlisting}

W podanej funkcji, inicjalizowany jest adapter TCP/IP oraz Wi-Fi. W pierwszej kolejności
tworzona jest struktura typu \verb+wifi_init_config_t+, którą inicjalizowany jest moduł.
W dalszej kolejności, tworzona jest struktura \verb+wifi_config_t+, która przechowuje 
informacje o sieci z którą moduł będzie się łączył. Jako wartości zmiennych \verb+.ssid+ 
oraz \verb+.password+ zostały ustawione wartości makr \verb+WIFI_SSID+ i \verb+WIFI_PASS+.\\

Ostatnim krokiem, jest ustawienie trybu pracy za pomocą funkcji \verb+esp_wifi_+
\verb+set_mode+ 
oraz załadowanie konfiguracji (funkcja \verb+esp_wifi_set_config+). Po poprawnej konfiguracji,
nie pozostaje nic innego jak tylko włączyć moduł Wi-Fi za pomocą polecenia \verb+esp_wifi_start+.


\subsection{Konfiguracja kanału PWM}
\label{projekt_pwm}
Jako ostatni, skofigurowany został kanał PWM, do sterowania jasnością diody LED.
W tym celu przygotowana została funkcja \verb+init_pwm+, która przygotowywała stosowny 
kanał.\\

\begin{lstlisting}[style=customc,
    frame=single,
    caption={Konfiguracja pojedynczego kanału PWM},
    captionpos=b,
    label={projekt_pwm}]
void init_pwm(){
  const int PWM_PERIOD = 500;
  const uint32_t pin_num = 14;
  uint32_t duty = 250; // realDuty = duty/PWM_PERIOD
  const int16_t phase = 0;
   
  pwm_init(PWM_PERIOD, &duty, 1, &pin_num);
  pwm_set_phases(&phase);
  pwm_start();
}
\end{lstlisting}

Inicjalizacją kanałów PWM zajmuje się funkcja systemowa \verb+pwm_init+, która jako argumenty
przyjmuje wartość okresu PWM, wypełnienia, ilość kanałów oraz numery pinów. W tym przypadku zdecydowałem się 
na okres $\num{500}$ \si{\micro s}, co daje częstotliwość $\num{2}$ \si{KHz}.
Warto zauważyć że funkcja przyjmuje wskaźniki na tablicę wypełnień oraz numerów pinów.
Pozwala to na inicjalizację wielu pinów jednym poleceniem. W tym przypadku, gdy interesujący
jest tylko jeden kanał, należy zamiast wartości zmiennych podać ich adresy.\\

Każda zmiana wypełnienia czy fazy generowanego sygnału, musi zostać zatwierdzona
poprzez wywołanie funkcji \verb+pwm_start+.

\section{Zadania w systemie czasu rzeczywistego FreeRTOS}
\label{projekt_taski}

W celu lepszego zapoznania się z systemem FreeRTOS, zdecydowałem się na rozdzielenie 
funkcjonalności na dwa zadania wykonujące się współbieżnie na jednym procesorze. 
Pierwsze zadanie \verb+sensor_task+ zajmuje się mierzeniem temperatury 
za pomocą konwertera analogowo-cyfrowego i alarmowaniem o za wysokiej temperaturze
za pomocą diody LED sterowanej sygnałem PWM. 
Drugie zadanie \verb+poster_task+,
zajmuje się komunikacją z serwerem poprzez REST API. Przesyła ono dane z czujnika
za pomocą metody HTTP POST. Zadania korzystają z współdzielonej danej, jaką jest 
wartość temperatury, dlatego też dostęp do tej zmiennej jest chroniony poprzez 
semafor binarny.

\subsection{Zadanie mierzenia temperatury}
\label{projekt_sensor_task}
Zadanie pomiarowe było realizowane przez funkcję \verb+adc_task+. Jej postać została przedstawiona 
na listingu \ref{projekt_adc_task}

\begin{lstlisting}[style=customc,
  frame=single,
  caption={Postać funkcji mierzącej temperaturę},
  captionpos=b,
  label={projekt_adc_task}]
void adc_task(void *pvParameters){
  const float A = 0.3101;
  const float B = -250.243;
  uint16_t adc_data[100];

  while (1) {
    if (ESP_OK == adc_read(&adc_data[0])) {
      ESP_LOGI(TAG, "adc read: %d", adc_data[0]);
      xSemaphoreTake(xMutex, portMAX_DELAY);
      sensorValue =  adc_data[0]*A + B;
      xSemaphoreGive(xMutex);
      uint32_t duty = 0;        
            
      if (sensorValue > 25.0){
        duty = (sensorValue - 25.0)*50;
        if (duty > 500) duty = 500;
        ESP_LOGI(TAG, "DUTY = %d", duty);
      }
      pwm_set_duties(&duty);
      pwm_start();
    }
    vTaskDelay(100 / portTICK_RATE_MS);
  }
}
\end{lstlisting}

W pierwszej kolejności, mierzone jest napięcie na termistorze za pomocą konwertera 
analogowo-cyfrowego, a następnie zamieniane na wartość temperatury wykorzystując
interpolację liniową. Zapis do zmiennej \verb+sensorValue+ odbywa się przy użyciu
semaforów binarnych, ponieważ z tej zmiennej korzystać będzie drugie zadanie. \\

Po dokonaniu pomiaru, zadanie sprawdza czy zmierzona temperatura nie jest większa od 
ustalonej wartości progowej od której moduł ma alarmować użytkownika o za wysokiej
temperaturze. Ustalono wartość progu równą $\num{25.0}$ stopni Celcjusza.\\

Wartość temperatury przekraczającej próg jest skalowana do wartości wypełnienia.
W przypadku przekroczenia wartości maksymalnej równej okresowi PWM, wartość wypełnienia 
nasyca się.\\

\subsection{Zadanie komunikacji z serwerem}
\label{projekt_poster_task}
Drugie zadanie obsługuje komunikację z serwerem, wykorzystując komunikację przez Wi-Fi.
Zadanie to jest wykonywane przez funkcję \verb+poster_task+. Funkcja ta jest o wiele bardziej
złożona niż poprzednie zadanie. Komunikacja z serwerem HTTP wymaga kilku czynności: \\

\begin{itemize}
    \item stworzenie żądania HTTP POST
    \item wykorzystanie usługi DNS do znalezienia adresu IP serwera
    \item utworzenie socketu
    \item połączenie się z socketem
    \item zapisanie wiadomości do socketu
    \item odebranie wiadomości zwrotnej
    \item zamknięcie socketu
\end{itemize}

W pierwszej kolejności tworzona jest wiadomość HTTP POST. Przykładowa postać tej
wiadomości została przedstawiona poniżej:

\begin{lstlisting}[style=customlatex,
    frame=single,
    caption={Przykładowa postać żądania HTTP POST},
    captionpos=b,
    label={projekt_adc_task}]
POST /data HTTP/1.0
Host: dell:8081
User-Agent: esp-idf/1.0 esp32"
Content-Type: application/x-www-form-urlencoded
Content-Length: 8

data=23.88

\end{lstlisting}

Wiadomość ta zostaje wygenerowana po odczytaniu wartości zmierzonej przez moduł.
Odczyt tej wartości odbywa się w sposób synchronizowany przez semafor binarny.\\

\subsubsection{Ciekawostka na temat funkcji \texttt{sprintf}}
W trakcie tworzenia oprogramowania, nastąpiła potrzeba zamiany wartości zmiennoprzecinkowej
na napis, tak aby mógł zostać on w prosty sposób dołączony do wysyłanej wiadomości.
Okazało się że funkcja \verb+sprintf+, która służy między innymi do przeprowadzania takiej
konwersji, nie obsługuje poprawnie formatu \verb+%f+. Jest to znany problem, opisywany
szeroko na forach internetowych. W celu zamiany liczby zmiennoprzecinkowej na napis
napisana została funkcja \verb+convertToFloat+, która obchodzi problem w dość prosty
ale mało efektywny sposób, dodatkowo obcinając dokładność liczby (zakładane są tylko 
dwie liczby po przecinku).\\

\subsubsection{Serwer danych}
Odebraniem wiadomości na komputerze PC, zajmuje się aplikacja napisana w języku 
Javascript, z wykorzystaniem biblioteki \verb+express+. Aplikacja ta odczytuje 
wartość otrzymanej temperatury i wyświetla ją na terminalu.

\begin{lstlisting}[style=customjs,
  frame=single,
  caption={Prosta aplikacja do zbierania danych z czujnika},
  captionpos=b,
  label={kod_serwera}]
var express = require('express');
var bodyParser = require('body-parser');
var app  = express();

app.use(bodyParser.urlencoded({ extended: false }));
app.use(bodyParser.json());

app.post('/data', (req, res) => {
    var data = req.body.data;
    console.log("Got " + data + " from ESP8266");
    res.send("OK\n");
});

var server = app.listen(8081, () => {
    var host = server.address().address;
    var port = server.address().port;

    console.log("ZAP Server listening at http://%s:%s", host, port)
})

\end{lstlisting}

\chapter{Przegląd pozostałych możliwości wykorzystania modułu}
\label{inne}

W tym rozdziale przedstawione zostały inne możliwości programowania 
modułu ESP8266. Język C ma to do siebie że jest językiem pozostającym 
blisko sprzętu, co może uprzykrzać życie hobbystom lub początkującym
konstruktorom urządzeń internetu rzeczy. Aby ułatwić programowanie, 
powstał szereg różnych wysokopoziomowych rozwiązań znacząco upraszczające 
korzystanie z modułu.

\section{Wykorzystanie języka skryptowego Lua}
\label{lua}
Po wgraniu specjalnego programu NodeMCU do pamięci Flash modułu,
możemy programować w języku skryptowym Lua. Programowanie odbywa się wtedy
z poziomu Arduino IDE.\\

Skrypty mogą być wgrywane jako pliki \verb+.lua+ lub pisane i wykonywane na bieżąco
tak jak na przykład w przypadku interpretera języka Python. Język Lua cechuje się 
olbrzymią prostotą i szybkością programowania. Przykładowo połączenie się z wifi 
wymaga dwóch linijek kodu:\\

\begin{lstlisting}[style=customjs,
    frame=single,
    caption={Kod łączący się z siecią Wi-Fi napisany w języku Lua},
    captionpos=b,
    label={lua_example}]
wifi.setmode(wifi.STATION)
wifi.sta.config("SSID","HASLO")
\end{lstlisting}

Prostota języka zachęca do korzystania z niego przy pisaniu aplikacji internetu 
rzeczy. Niestety jako język wysokopoziomowy, jest mocno oddzielony od sprzętu i nie 
pozwala na zrozumienie architektury modułu. Po resecie urządzenia, oprogramowanie szuka
zapisanego pliku \texttt{init.lua} z programem użytkownika.

\section{Wykorzystanie MicroPythona}
\label{micropython}
Korzystanie z MicroPythona jest bardzo podobne do korzystania z NodeMCU. W pierwszej
kolejności należy wgrać interpreter na moduł ESP8266. Po resecie, oprogramowanie szuka
pliku \texttt{main.py} z programem użytkownika.
Język programowania wykorzystywany w MicroPythonie różni się nieznacznie od zwykłego
Pythona w wersji 3. Różnice są jednak niewielkie, dlatego jeżeli ktoś zna Pythona to
bardzo łatwo jest mu zaprogramować urządzenie. Poniżej znajduje się kod 
do połączenia modułu z siecią Wi-Fi.

\begin{lstlisting}[style=custompython, caption={Przykładowy kod do 
    połączenia się z siecią Wi-Fi w języku MicroPython},
    captionpos=b]
import network
wlan = network.WLAN(network.STA_IF)
wlan.active(True)
if not wlan.isconnected():
    print('connecting to network...')
    wlan.connect('essid', 'password')
    while not wlan.isconnected():
        pass
\end{lstlisting}


 
\section{AT Commands}
\label{AT}
Aby wykorzystać moduł ESP8266 do połaczenia się z siecią Wi-Fi, nie jest wymagana
umiejętność programowania urządzenia. Większość urządzeń z pudełka przychodzi z wgranym
programem ESP8266 AT Firmware, który pozwala na łączenie sie modułu z siecią Wi-Firmware
za pomocą specjalnych komend AT. Zasada korzystania z modułu jest prosta. Urządzenie 
nadrzędne wysyła poprzez port szeregowy komendę a następnie przez port szeregowy odbiera dane.
Moduł ESP8266 pełni rolę pośrednika w komunikacji. Na poniższym listingu znajduje się 
kod na Arduino, który uruchamia serwer HTTP za pośrednictwem modułu.\\

\begin{lstlisting}[style=customc,
    frame=single,
    caption={Kod uruchamiający serwer HTTP na płytce Arduino z wykorzystaniem modułu ESP8266},
    captionpos=b,
    label={projekt_adc_task}]
sendData("AT+RST\r\n",500,DEBUG); 
sendData("AT+CWMODE=2\r\n",500,DEBUG); 
sendData("AT+CIFSR\r\n",500,DEBUG);
sendData("AT+CIPMUX=1\r\n",500,DEBUG); 
sendData("AT+CIPSERVER=1,80\r\n",500,DEBUG); 
\end{lstlisting}

W pierwszej kolejności urządzenie nadrzędne resetuje moduł i przestawia go w tryb 
\textit{Access Point}. Następnie konfiguruje moduł tak aby uzyskał adres IP oraz 
umożliwił połączenie wielu stacjom. Ostatnia komenda uruchamia serwer.



\begin{thebibliography}{1}
    \bibitem{datasheet}
    Espressif Systems 
    \textit{ESP8266 Datasheet}.\\
    \texttt{https://bbs.espressif.com}, 2015.\\

    \bibitem{reference}
    Espressif Systems
    \textit{ESP8266 Technical Reference}.\\
    \texttt{https://bbs.espressif.com}, 2017.\\

    \bibitem{kolban}
    Neil Kolban.
    \textit{Kolban's Book on the ESP32 \& ESP8266}. \\
    \texttt{https://www.neilkolban.com/tech/}, Texas, USA, 2016.\\

    \bibitem{olimex}
    OLIMEX \textsuperscript{\textcopyright}.
    \textit{How to use ESP8266 with Arduino IDE}.\\
    \texttt{https://www.olimex.com/}, 2017.\\

    \bibitem{at_instruction_set}
    Espressif Systems.
    \textit{ESP8266. AT Instruction Set}.\\
    \texttt{https://www.espressif.com}, 2019.\\

    \bibitem{espressif_guide}
    Espressif Systems.
    \textit{ESP8266 SDK. Getting Started Guide}.\\
    \texttt{https://www.espressif.com}, 2019.\\

\end{thebibliography}




\end{document}