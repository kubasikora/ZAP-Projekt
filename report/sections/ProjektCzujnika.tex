\chapter{Projekt czujnika opartego o moduł ESP8266 napisany w języku C}
\label{projekt}

W celu prezentacji architektury modułu ESP8266 zdecydowałem się zaprogramować 
moduł w taki sposób aby mierzył on temperaturę i wysyłał tą informację do serwera
poprzez sieć Wi-Fi. Temperatura będzie mierzona analogowo za pomocą termistora, 
podłączonego do konwertera analogowo-cyfrowego.

\section{Konfiguracja}
\label{projekt_konfiguracja}

\subsection{Interfejs szeregowy UART}
\label{projekt_uart}
W celu ułatwienia dalszej pracy, pracę na projektem rozpocząłem od konfiguracji
interfejsu UART, tak aby móc w prosty sposób wyswietlać na ekranie swojego komputera
informacje pochodzące od modułu. Inicjalizacją interfejsu UART zajmuje się funkcja
\verb+uart_init()+. \\

\begin{lstlisting}[style=customc,
    frame=single,
    caption={Konfiguracja interfejsu UART},
    captionpos=b,
    label={esptool_basic}]
void uart_init(){
  uart_config_t uart_config = {
    .baud_rate = 115200,
    .data_bits = UART_DATA_8_BITS,
    .parity    = UART_PARITY_DISABLE,
    .stop_bits = UART_STOP_BITS_1,
    .flow_ctrl = UART_HW_FLOWCTRL_DISABLE
  };
  uart_param_config(UART_NUM_0, &uart_config);
  uart_driver_install(UART_NUM_0, ECHO_BUFFER_SIZE*2, 
                      0, 0, NULL);
}
\end{lstlisting}

Funkcja ta w typowy sposób, znany z mikrokontrolerów z rodziny STM32, inicjuje interfejs UART
przy użyciu struktury przechowującej poszczególne parametry. W tym przypadku, struktura ta 
jest typu \verb+uart_config_t+. W niej zostały ustawione:\\
\begin{itemize}
    \item prędkość transmisji 
    \item ilość bitów danych
    \item bit parzystości
    \item ilość bitów stopu
    \item kontrola transmisji\\
\end{itemize} 

Rzeczywista konfiguracja interfejsu następuje po wywołaniu po sobie funkcji \verb+uart_param_config+ oraz
\verb+uart_driver_install+ z odpowiednim numerem interfejsu (w tym przypadku \verb+UART_NUM_0+).
\newpage

\subsection{Konwerter analogowo-cyfrowy}
\label{projekt_adc}

W podobny sposób został skonfigurowany konwerter analogo-cyfrowy do pomiaru napięcia odkładającego
się na termistorze. W tym celu przygotowana została funkcja \verb+initialize_adc+, która przygotowuje 
konwerter do pracy.

\begin{lstlisting}[style=customc,
    frame=single,
    caption={Konfiguracja ADC},
    captionpos=b,
    label={esptool_basic}]
void initialize_adc(){
  adc_config_t adc_config;
  adc_config.mode = ADC_READ_TOUT_MODE;
  adc_config.clk_div = 8;
  ESP_ERROR_CHECK(adc_init(&adc_config));
}
\end{lstlisting}

W tym przypadku, przygotowana została struktura typu \verb+adc_config_t+,
 w której ustawione zostały 
dwa parametry: tryb pracy oraz dzielnik częstotliwości taktowania konwertera. 
Tryb \verb+ADC_READ_TOUT_MODE+
ustawia konwerter w tryb pomiaru napięcia zewnętrznego. Rzeczywista 
konfiguracja konwertera następuję
w chwili wywołania funkcji \verb+adc_init+. Wywołanie tej funkcji 
makrem, pozwala na wypisanie informacji
o błędzie w przypadku niepowodzenia konfiguracji konwertera.


\subsection{Wi-Fi}
\label{projekt_wifi}

Zdecydowanie najciekawszym elementem do skonfigurowania była część 
obsługująca komunikację przez sieć Wi-Fi. W tym celu przygotowana została
funkcja \verb+wifi_init_sta+, która konfiguruje moduł w trybie \textit{Station}.\\

\begin{lstlisting}[style=customc,
    frame=single,
    caption={Konfiguracja Wi-Fi},
    captionpos=b,
    label={esptool_basic}]
void wifi_init_sta(){
  wifi_event_group = xEventGroupCreate();
  tcpip_adapter_init();

  wifi_init_config_t cfg = WIFI_INIT_CONFIG_DEFAULT();
  ESP_ERROR_CHECK(esp_wifi_init(&cfg));
  wifi_config_t wifi_config = {
    .sta = {
      .ssid = WIFI_SSID,
      .password = WIFI_PASS
    },
  };

  ESP_ERROR_CHECK(esp_wifi_set_mode(WIFI_MODE_STA));
  ESP_ERROR_CHECK(esp_wifi_set_config(ESP_IF_WIFI_STA, 
                                      &wifi_config));
  ESP_ERROR_CHECK(esp_wifi_start());
}
\end{lstlisting}

W podanej funkcji, inicjalizowany jest adapter TCP/IP oraz Wi-Fi. W pierwszej kolejności
tworzona jest struktura typu \verb+wifi_init_config_t+, którą inicjalizowany jest moduł.
W dalszej kolejności, tworzona jest struktura \verb+wifi_config_t+, która przechowuje 
informacje o sieci z którą moduł będzie się łączył. Jako wartości zmiennych \verb+.ssid+ 
oraz \verb+.password+ zostały ustawione wartości makr \verb+WIFI_SSID+ i \verb+WIFI_PASS+.\\

Ostatnim krokiem, jest ustawienie trybu pracy za pomocą funkcji \verb+esp_wifi_+
\verb+set_mode+ 
oraz załadowanie konfiguracji (funkcja \verb+esp_wifi_set_config+). Po poprawnej konfiguracji,
nie pozostaje nic innego jak tylko włączyć moduł Wi-Fi za pomocą polecenia \verb+esp_wifi_start+.


\section{Zadania w systemie czasu rzeczywistego FreeRTOS}
\label{projekt_taski}

W celu lepszego zapoznania się z systemem FreeRTOS, zdecydowałem się na rozdzielenie 
funkcjonalności na dwa zadania wykonujące się współbieżnie na jednym procesorze. 
Pierwsze zadanie \verb+sensor_task+ zajmowało się zapisa

\section{}
