\chapter{Projekt czujnika opartego o moduł ESP8266 napisany w języku C}
\label{projekt}

W celu prezentacji architektury modułu ESP8266 zdecydowałem się zaprogramować 
moduł w taki sposób aby mierzył on temperaturę i wysyłał tą informację do serwera
poprzez sieć Wi-Fi. Temperatura będzie mierzona analogowo za pomocą termistora, 
podłączonego do konwertera analogowo-cyfrowego. Czujnik będzie sygnalizował 
temperaturę za pomocą diody LED, której jasność będzie zwiększała się wraz z 
odczytem czujnika.

\section{Konfiguracja}
\label{projekt_konfiguracja}

\subsection{Interfejs szeregowy UART}
\label{projekt_uart}
W celu ułatwienia dalszej pracy, pracę na projektem rozpocząłem od konfiguracji
interfejsu UART, tak aby móc w prosty sposób wyswietlać na ekranie swojego komputera
informacje pochodzące od modułu. Inicjalizacją interfejsu UART zajmuje się funkcja
\verb+uart_init()+. \\

\begin{lstlisting}[style=customc,
    frame=single,
    caption={Konfiguracja interfejsu UART},
    captionpos=b,
    label={esptool_basic}]
void uart_init(){
  uart_config_t uart_config = {
    .baud_rate = 115200,
    .data_bits = UART_DATA_8_BITS,
    .parity    = UART_PARITY_DISABLE,
    .stop_bits = UART_STOP_BITS_1,
    .flow_ctrl = UART_HW_FLOWCTRL_DISABLE
  };
  uart_param_config(UART_NUM_0, &uart_config);
  uart_driver_install(UART_NUM_0, ECHO_BUFFER_SIZE*2, 
                      0, 0, NULL);
}
\end{lstlisting}

Funkcja ta w typowy sposób, znany z mikrokontrolerów z rodziny STM32, inicjuje interfejs UART
przy użyciu struktury przechowującej poszczególne parametry. W tym przypadku, struktura ta 
jest typu \verb+uart_config_t+. W niej zostały ustawione:\\
\begin{itemize}
    \item prędkość transmisji 
    \item ilość bitów danych
    \item bit parzystości
    \item ilość bitów stopu
    \item kontrola transmisji\\
\end{itemize} 

Rzeczywista konfiguracja interfejsu następuje po wywołaniu po sobie funkcji \verb+uart_param_config+ oraz
\verb+uart_driver_install+ z odpowiednim numerem interfejsu (w tym przypadku \verb+UART_NUM_0+).
\newpage

\subsection{Konwerter analogowo-cyfrowy}
\label{projekt_adc}

W podobny sposób został skonfigurowany konwerter analogo-cyfrowy do pomiaru napięcia odkładającego
się na termistorze. W tym celu przygotowana została funkcja \verb+initialize_adc+, która przygotowuje 
konwerter do pracy.

\begin{lstlisting}[style=customc,
    frame=single,
    caption={Konfiguracja ADC},
    captionpos=b,
    label={esptool_basic}]
void initialize_adc(){
  adc_config_t adc_config;
  adc_config.mode = ADC_READ_TOUT_MODE;
  adc_config.clk_div = 8;
  ESP_ERROR_CHECK(adc_init(&adc_config));
}
\end{lstlisting}

W tym przypadku, przygotowana została struktura typu \verb+adc_config_t+,
 w której ustawione zostały 
dwa parametry: tryb pracy oraz dzielnik częstotliwości taktowania konwertera. 
Tryb \verb+ADC_READ_TOUT_MODE+
ustawia konwerter w tryb pomiaru napięcia zewnętrznego. Rzeczywista 
konfiguracja konwertera następuję
w chwili wywołania funkcji \verb+adc_init+. Wywołanie tej funkcji 
makrem, pozwala na wypisanie informacji
o błędzie w przypadku niepowodzenia konfiguracji konwertera.


\subsection{Wi-Fi}
\label{projekt_wifi}

Zdecydowanie najciekawszym elementem do skonfigurowania była część 
obsługująca komunikację przez sieć Wi-Fi. W tym celu przygotowana została
funkcja \verb+wifi_init_sta+, która konfiguruje moduł w trybie \textit{Station}.\\

\begin{lstlisting}[style=customc,
    frame=single,
    caption={Konfiguracja Wi-Fi},
    captionpos=b,
    label={esptool_basic}]
void wifi_init_sta(){
  wifi_event_group = xEventGroupCreate();
  tcpip_adapter_init();

  wifi_init_config_t cfg = WIFI_INIT_CONFIG_DEFAULT();
  ESP_ERROR_CHECK(esp_wifi_init(&cfg));
  wifi_config_t wifi_config = {
    .sta = {
      .ssid = WIFI_SSID,
      .password = WIFI_PASS
    },
  };

  ESP_ERROR_CHECK(esp_wifi_set_mode(WIFI_MODE_STA));
  ESP_ERROR_CHECK(esp_wifi_set_config(ESP_IF_WIFI_STA, 
                                      &wifi_config));
  ESP_ERROR_CHECK(esp_wifi_start());
}
\end{lstlisting}

W podanej funkcji, inicjalizowany jest adapter TCP/IP oraz Wi-Fi. W pierwszej kolejności
tworzona jest struktura typu \verb+wifi_init_config_t+, którą inicjalizowany jest moduł.
W dalszej kolejności, tworzona jest struktura \verb+wifi_config_t+, która przechowuje 
informacje o sieci z którą moduł będzie się łączył. Jako wartości zmiennych \verb+.ssid+ 
oraz \verb+.password+ zostały ustawione wartości makr \verb+WIFI_SSID+ i \verb+WIFI_PASS+.\\

Ostatnim krokiem, jest ustawienie trybu pracy za pomocą funkcji \verb+esp_wifi_+
\verb+set_mode+ 
oraz załadowanie konfiguracji (funkcja \verb+esp_wifi_set_config+). Po poprawnej konfiguracji,
nie pozostaje nic innego jak tylko włączyć moduł Wi-Fi za pomocą polecenia \verb+esp_wifi_start+.


\subsection{Konfiguracja kanału PWM}
\label{projekt_pwm}
Jako ostatni, skofigurowany został kanał PWM, do sterowania jasnością diody LED.
W tym celu przygotowana została funkcja \verb+init_pwm+, która przygotowywała stosowny 
kanał.\\

\begin{lstlisting}[style=customc,
    frame=single,
    caption={Konfiguracja pojedynczego kanału PWM},
    captionpos=b,
    label={projekt_pwm}]
void init_pwm(){
  const int PWM_PERIOD = 500;
  const uint32_t pin_num = 14;
  uint32_t duty = 250; // realDuty = duty/PWM_PERIOD
  const int16_t phase = 0;
   
  pwm_init(PWM_PERIOD, &duty, 1, &pin_num);
  pwm_set_phases(&phase);
  pwm_start();
}
\end{lstlisting}

Inicjalizacją kanałów PWM zajmuje się funkcja systemowa \verb+pwm_init+, która jako argumenty
przyjmuje wartość okresu PWM, wypełnienia, ilość kanałów oraz numery pinów. W tym przypadku zdecydowałem się 
na okres $\num{500}$ \si{\micro s}, co daje częstotliwość $\num{2}$ \si{KHz}.
Warto zauważyć że funkcja przyjmuje wskaźniki na tablicę wypełnień oraz numerów pinów.
Pozwala to na inicjalizację wielu pinów jednym poleceniem. W tym przypadku, gdy interesujący
jest tylko jeden kanał, należy zamiast wartości zmiennych podać ich adresy.\\

Każda zmiana wypełnienia czy fazy generowanego sygnału, musi zostać zatwierdzona
poprzez wywołanie funkcji \verb+pwm_start+.

\section{Zadania w systemie czasu rzeczywistego FreeRTOS}
\label{projekt_taski}

W celu lepszego zapoznania się z systemem FreeRTOS, zdecydowałem się na rozdzielenie 
funkcjonalności na dwa zadania wykonujące się współbieżnie na jednym procesorze. 
Pierwsze zadanie \verb+sensor_task+ zajmuje się mierzeniem temperatury 
za pomocą konwertera analogowo-cyfrowego i alarmowaniem o za wysokiej temperaturze
za pomocą diody LED sterowanej sygnałem PWM. 
Drugie zadanie \verb+poster_task+,
zajmuje się komunikacją z serwerem poprzez REST API. Przesyła ono dane z czujnika
za pomocą metody HTTP POST. Zadania korzystają z współdzielonej danej, jaką jest 
wartość temperatury, dlatego też dostęp do tej zmiennej jest chroniony poprzez 
semafor binarny.

\subsection{Zadanie mierzenia temperatury}
\label{projekt_sensor_task}
Zadanie pomiarowe było realizowane przez funkcję \verb+adc_task+. Jej postać została przedstawiona 
na listingu \ref{projekt_adc_task}

\begin{lstlisting}[style=customc,
  frame=single,
  caption={Postać funkcji mierzącej temperaturę},
  captionpos=b,
  label={projekt_adc_task}]
void adc_task(void *pvParameters){
  const float A = 0.3101;
  const float B = -250.243;
  uint16_t adc_data[100];

  while (1) {
    if (ESP_OK == adc_read(&adc_data[0])) {
      ESP_LOGI(TAG, "adc read: %d", adc_data[0]);
      xSemaphoreTake(xMutex, portMAX_DELAY);
      sensorValue =  adc_data[0]*A + B;
      xSemaphoreGive(xMutex);
      uint32_t duty = 0;        
            
      if (sensorValue > 25.0){
        duty = (sensorValue - 25.0)*50;
        if (duty > 500) duty = 500;
        ESP_LOGI(TAG, "DUTY = %d", duty);
      }
      pwm_set_duties(&duty);
      pwm_start();
    }
    vTaskDelay(100 / portTICK_RATE_MS);
  }
}
\end{lstlisting}

W pierwszej kolejności, mierzone jest napięcie na termistorze za pomocą konwertera 
analogowo-cyfrowego, a następnie zamieniane na wartość temperatury wykorzystując
interpolację liniową. Zapis do zmiennej \verb+sensorValue+ odbywa się przy użyciu
semaforów binarnych, ponieważ z tej zmiennej korzystać będzie drugie zadanie. \\

Po dokonaniu pomiaru, zadanie sprawdza czy zmierzona temperatura nie jest większa od 
ustalonej wartości progowej od której moduł ma alarmować użytkownika o za wysokiej
temperaturze. Ustalono wartość progu równą $\num{25.0}$ stopni Celcjusza.\\

Wartość temperatury przekraczającej próg jest skalowana do wartości wypełnienia.
W przypadku przekroczenia wartości maksymalnej równej okresowi PWM, wartość wypełnienia 
nasyca się.\\

\subsection{Zadanie komunikacji z serwerem}
\label{projekt_poster_task}
Drugie zadanie obsługuje komunikację z serwerem, wykorzystując komunikację przez Wi-Fi.
Zadanie to jest wykonywane przez funkcję \verb+poster_task+. Funkcja ta jest o wiele bardziej
złożona niż poprzednie zadanie. Komunikacja z serwerem HTTP wymaga kilku czynności: \\

\begin{itemize}
    \item stworzenie żądania HTTP POST
    \item wykorzystanie usługi DNS do znalezienia adresu IP serwera
    \item utworzenie socketu
    \item połączenie się z socketem
    \item zapisanie wiadomości do socketu
    \item odebranie wiadomości zwrotnej
    \item zamknięcie socketu
\end{itemize}

W pierwszej kolejności tworzona jest wiadomość HTTP POST. Przykładowa postać tej
wiadomości została przedstawiona poniżej:

\begin{lstlisting}[style=customlatex,
    frame=single,
    caption={Przykładowa postać żądania HTTP POST},
    captionpos=b,
    label={projekt_adc_task}]
POST /data HTTP/1.0
Host: dell:8081
User-Agent: esp-idf/1.0 esp32"
Content-Type: application/x-www-form-urlencoded
Content-Length: 8

data=23.88

\end{lstlisting}

Wiadomość ta zostaje wygenerowana po odczytaniu wartości zmierzonej przez moduł.
Odczyt tej wartości odbywa się w sposób synchronizowany przez semafor binarny.\\

\subsubsection{Ciekawostka na temat funkcji \texttt{sprintf}}
W trakcie tworzenia oprogramowania, nastąpiła potrzeba zamiany wartości zmiennoprzecinkowej
na napis, tak aby mógł zostać on w prosty sposób dołączony do wysyłanej wiadomości.
Okazało się że funkcja \verb+sprintf+, która służy między innymi do przeprowadzania takiej
konwersji, nie obsługuje poprawnie formatu \verb+%f+. Jest to znany problem, opisywany
szeroko na forach internetowych. W celu zamiany liczby zmiennoprzecinkowej na napis
napisana została funkcja \verb+convertToFloat+, która obchodzi problem w dość prosty
ale mało efektywny sposób, dodatkowo obcinając dokładność liczby (zakładane są tylko 
dwie liczby po przecinku).\\

\subsubsection{Serwer danych}
Odebraniem wiadomości na komputerze PC, zajmuje się aplikacja napisana w języku 
Javascript, z wykorzystaniem biblioteki \verb+express+. Aplikacja ta odczytuje 
wartość otrzymanej temperatury i wyświetla ją na terminalu.

\begin{lstlisting}[style=customjs,
  frame=single,
  caption={Prosta aplikacja do zbierania danych z czujnika},
  captionpos=b,
  label={kod_serwera}]
var express = require('express');
var bodyParser = require('body-parser');
var app  = express();

app.use(bodyParser.urlencoded({ extended: false }));
app.use(bodyParser.json());

app.post('/data', (req, res) => {
    var data = req.body.data;
    console.log("Got " + data + " from ESP8266");
    res.send("OK\n");
});

var server = app.listen(8081, () => {
    var host = server.address().address;
    var port = server.address().port;

    console.log("ZAP Server listening at http://%s:%s", host, port)
})

\end{lstlisting}
