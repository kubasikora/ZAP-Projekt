\chapter{Przegląd pozostałych możliwości wykorzystania modułu}
\label{inne}

W tym rozdziale przedstawione zostały inne możliwości programowania 
modułu ESP8266. Język C ma to do siebie że jest językiem pozostającym 
blisko sprzętu, co może uprzykrzać życie hobbystom lub początkującym
konstruktorom urządzeń internetu rzeczy. Aby ułatwić programowanie, 
powstał szereg różnych wysokopoziomowych rozwiązań znacząco upraszczające 
korzystanie z modułu.

\section{Wykorzystanie języka skryptowego Lua}
\label{lua}
Po wgraniu specjalnego programu NodeMCU do pamięci Flash modułu,
możemy programować w języku skryptowym Lua. Programowanie odbywa się wtedy
z poziomu Arduino IDE.\\

Skrypty mogą być wgrywane jako pliki \verb+.lua+ lub pisane i wykonywane na bieżąco
tak jak na przykład w przypadku interpretera języka Python. Język Lua cechuje się 
olbrzymią prostotą i szybkością programowania. Przykładowo połączenie się z wifi 
wymaga dwóch linijek kodu:\\

\begin{lstlisting}[style=customjs,
    frame=single,
    caption={Kod łączący się z siecią Wi-Fi napisany w języku Lua},
    captionpos=b,
    label={lua_example}]
wifi.setmode(wifi.STATION)
wifi.sta.config("SSID","HASLO")
\end{lstlisting}

Prostota języka zachęca do korzystania z niego przy pisaniu aplikacji internetu 
rzeczy. Niestety jako język wysokopoziomowy, jest mocno oddzielony od sprzętu i nie 
pozwala na zrozumienie architektury modułu. Po resecie urządzenia, oprogramowanie szuka
zapisanego pliku \texttt{init.lua} z programem użytkownika.

\section{Wykorzystanie MicroPythona}
\label{micropython}
Korzystanie z MicroPythona jest bardzo podobne do korzystania z NodeMCU. W pierwszej
kolejności należy wgrać interpreter na moduł ESP8266. Po resecie, oprogramowanie szuka
pliku \texttt{main.py} z programem użytkownika.
Język programowania wykorzystywany w MicroPythonie różni się nieznacznie od zwykłego
Pythona w wersji 3. Różnice są jednak niewielkie, dlatego jeżeli ktoś zna Pythona to
bardzo łatwo jest mu zaprogramować urządzenie. Poniżej znajduje się kod 
do połączenia modułu z siecią Wi-Fi.

\begin{lstlisting}[style=custompython, caption={Przykładowy kod do 
    połączenia się z siecią Wi-Fi w języku MicroPython},
    captionpos=b]
import network
wlan = network.WLAN(network.STA_IF)
wlan.active(True)
if not wlan.isconnected():
    print('connecting to network...')
    wlan.connect('essid', 'password')
    while not wlan.isconnected():
        pass
\end{lstlisting}


 
\section{AT Commands}
\label{AT}
Aby wykorzystać moduł ESP8266 do połaczenia się z siecią Wi-Fi, nie jest wymagana
umiejętność programowania urządzenia. Większość urządzeń z pudełka przychodzi z wgranym
programem ESP8266 AT Firmware, który pozwala na łączenie sie modułu z siecią Wi-Firmware
za pomocą specjalnych komend AT. Zasada korzystania z modułu jest prosta. Urządzenie 
nadrzędne wysyła poprzez port szeregowy komendę a następnie przez port szeregowy odbiera dane.
Moduł ESP8266 pełni rolę pośrednika w komunikacji. Na poniższym listingu znajduje się 
kod na Arduino, który uruchamia serwer HTTP za pośrednictwem modułu.\\

\begin{lstlisting}[style=customc,
    frame=single,
    caption={Kod uruchamiający serwer HTTP na płytce Arduino z wykorzystaniem modułu ESP8266},
    captionpos=b,
    label={projekt_adc_task}]
sendData("AT+RST\r\n",500,DEBUG); 
sendData("AT+CWMODE=2\r\n",500,DEBUG); 
sendData("AT+CIFSR\r\n",500,DEBUG);
sendData("AT+CIPMUX=1\r\n",500,DEBUG); 
sendData("AT+CIPSERVER=1,80\r\n",500,DEBUG); 
\end{lstlisting}

W pierwszej kolejności urządzenie nadrzędne resetuje moduł i przestawia go w tryb 
\textit{Access Point}. Następnie konfiguruje moduł tak aby uzyskał adres IP oraz 
umożliwił połączenie wielu stacjom. Ostatnia komenda uruchamia serwer.


